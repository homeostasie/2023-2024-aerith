\input{../doc-class-bb.tex}


\begin{document}


\begin{center} % Center everything on the page

\textsc{\LARGE Collège Faubert}\\

\horrule{2px}

\textsc{\large Devoir Commun}\\% Title of your document
\textsc{\large Mathématiques}\\% Title of your document
\textsc{\large 13 Novembre 2023}\\

\horrule{2px}

\begin{itemize}[label={$\bullet$}]
  \item \textsc{Exercice 1} - 15 points     
  \item \textsc{Exercice 2} - 10 points 
  \item \textsc{Exercice 3} - 14 points 
  \item \textsc{Exercice 4} - 11 points 
\end{itemize}

\horrule{1px}

\begin{itemize}
  \item L'usage de la calculatrice de type collège est autorisé.
  \item L'usage de tout autre document est interdit. 
\end{itemize}

\horrule{2px}

\end{center} 


\subsection*{Exercice 1 - 15 points}

\parbox{0.55\linewidth}{Sur le dessin ci-contre : 
\begin{itemize}[label={$\bullet$}]
    \item Les points A, B et E sont alignés, et C le milieu de [BD].
    \item AC = 5cm, AB = 4cm, CD = 3cm et BE = 7cm.
  \end{itemize}

  \vspace{0.5cm}

\begin{enumerate}
    \item Quelle est la nature du triangle ABC ? Justifier. \\
    \item En déduire la nature du triangle BDE. \\
    \item Calculer ED. Arrondir le résultat au dixième. 
\end{enumerate}}\hfill
\parbox{0.42\linewidth}{\psset{unit=0.8cm}
\begin{pspicture}(6.25,4)
    \pspolygon(2.4,1.8)(0.4,0.3)(2.4,0.3)(2.4,3.3)(6,0.3)(2.4,0.3)%CABDE
    \uput[l](2.4,2.55){3cm}\uput[ul](1.6,1){5cm}\uput[d](1.4,0.3){4cm}
    \uput[d](4.6,0.3){7cm}
    \psdots(2.4,1.8)(0.4,0.3)(2.4,0.3)(2.4,3.3)(6,0.3)(2.4,0.3)
    \uput[dl](0.4,0.3){A} \uput[d](2.4,0.3){B} \uput[ul](2.4,1.8){C} \uput[u](2.4,3.3){D} \uput[dr](6,0.3){E} 
\end{pspicture}}

\medskip

\subsection*{Exercice 2 - 10 points}

\medskip

\begin{enumerate}
    \item Calculer $5x^2 - 3(2x+1)$ pour $x = 4$. \\
    \item Montrer que, pour toute valeur de $x$, on a: $5x^2 - 3(2x + 1) = 5x^2 - 6x - 3$. \\
    \item Trouver la valeur de $x$ pour laquelle $5x^2 - 3(2x+1)= 5x^2 - 4x +1$.
\end{enumerate}

\medskip

\subsection*{Exercice 3 - 14 points}

\medskip

On considère le programme de calcul ci-dessous :

\parbox{0.35\linewidth}{
\fbox{\begin{minipage}{0.35\textwidth}
\begin{itemize}[label={$\bullet$}]
    \item choisir un nombre de départ
    \item multiplier ce nombre par $(- 2)$
    \item ajouter $5$ au produit
    \item multiplier le résultat par $5$
    \item écrire le résultat obtenu.
\end{itemize}\end{minipage}}
}
\hfill
\parbox{0.64\linewidth}{
\begin{enumerate}
    \item 
        \begin{enumerate}
            \item Vérifier que, lorsque le nombre de départ est $2$, on obtient $5$. 
            \item Lorsque le nombre de départ est $3$, quel résultat obtient-on ? \\
        \end{enumerate}
    \item Quel nombre faut-il choisir au départ pour que le résultat soit $0$ ? \\
    \item Arthur prétend que, pour n'importe quel nombre de départ $x$, l'expression $(x - 5)^2 - x^2$ permet d'obtenir le résultat du programme de calcul.
        
    A-t-il raison ?
\end{enumerate}}

\newpage

\subsection*{Exercice 4 - 11 points}

\medskip

Une entreprise rembourse à ses employés le coût de leurs déplacements professionnels, quand les employés utilisent leur véhicule personnel.\\

Pour calculer le montant de ces remboursements, elle utilise la formule et le tableau d'équivalence ci-dessous proposés par le gestionnaire:

\begin{center}
\begin{tabularx}{\linewidth}{|p{5cm}|X|}\hline
\multicolumn{2}{|c|}{\textbf{Document 1}}\\
Formule 		&Tableau\\
Montant du remboursement:

\qquad $a + b\times d$

où :

$\bullet~~$ $a$ est un prix (en euros) qui ne dépend que de la longueur du trajet;

$\bullet~~$ $b$ est le prix payé (en euros) par kilomètre parcouru;

$\bullet~~$ $d$ est la longueur en kilomètres du \og trajet aller \fg.&
\begin{tabular}{|m{3cm}|c| c|}\hline
%\begin{tabularx}{\linewidth}{|m{3cm}|*{2}{>{\centering \arraybackslash}X|}}\hline
Longueur $d$ du \og trajet aller\fg	&Prix $a$&Prix $b$ par kilomètre\\ \hline
 De 1 km à 16 km						&\np{0,7781}		&\np{0,1944}\\ \hline
 De 17 km à 32 km 						&\np{0,2503}		&\np{0,2165}\\ \hline
 De 33 km à 64 km 						&\np{2,0706}		&\np{0,1597}\\ \hline
 De 65 km à 109 km 						&\np{2,8891}		&\np{0,1489}\\ \hline
 De 110 km à 149 km 					&\np{4,0864}		&\np{0,1425}\\ \hline
 De 150 km à 199 km 					&\np{8,0871}		&\np{0,1193}\\ \hline
 De 200 km à 300 km 					&\np{7,7577}		&\np{0,1209}\\ \hline
 De 301 km à 499 km 					&\np{13,6514}	&\np{0,1030}\\ \hline
 De 500 km à 799 km 					&\np{18,4449}	&\np{0,0921}\\ \hline
 De 800 km à \np{9999}~km				&\np{32,2041}	&\np{0,0755}\\ \hline
\end{tabular} \\ \hline
\end{tabularx}
\end{center}
\medskip

\begin{enumerate}
\item Pour un \og trajet aller\fg{} de 30~km, vérifier que le montant du remboursement est environ $6,75$~\euro. \\
\item  Dans le cadre de son travail, un employé de cette entreprise effectue un déplacement à Paris. Il choisit de prendre sa voiture et il trouve les informations ci-dessous sur un site internet.

\begin{center}
\begin{tabularx}{\linewidth}{|X|}\hline
\multicolumn{1}{|c|}{\textbf{Document 2}}\\
Distance Nantes - Paris : 386 km\\
Coût du péage entre Nantes et Paris: 37~\euro\\
Consommation moyenne de la voiture de l'employé: $6,2$ litres d'essence aux $100$~km\\
Prix du litre d'essence: 1,52~\euro\\\hline
\end{tabularx}
\end{center}

\textbf{À l'aide des documents 1 et 2, répondre à la question suivante:} \\

\og Le montant du remboursement sera-t-il suffisant pour couvrir les dépenses de cet employé pour effectuer le \og trajet aller\fg{} de Nantes à Paris ? \fg
\end{enumerate}

\end{document}
