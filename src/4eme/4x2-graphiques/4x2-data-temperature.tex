\input{../doc-class-cours.tex}

\begin{document}

Chaque période dure 30ans. Pour chacune des périodes, on a une donnée de température tous les 4ans.
Pour chaque période, représenter les données avec un diagramme en bâtons.

On impose l'échelle : 
\begin{itemize}
      \item ordonnée : graduée de $0 \degree C$ à $3 \degree C$ avec 8cm pour $1 \degree C$.
      \item abscisse : graduée avec $1cm = 2ans$.
\end{itemize}

\subsection*{Période 1 - 1897 - 1927}



$$Données : 1\degree C ; 0.8\degree C ; 0.6\degree C ; 0.7\degree C ;  0.6\degree C ; 0.7\degree C ; 1\degree C ; 1.1\degree C ; 0.9\degree C$$


\subsection*{Période 2 - 1927 - 1957}

$$Données : 0.9\degree C ; 1.1\degree C ; 1.2\degree C ; 1.3\degree C ;  1.4\degree C ; 1.4\degree C ; 1.2\degree C ; 1.1\degree C ; 1.\degree C$$



\subsection*{Période 3 - 1957 - 1987}

$$Données : 1\degree C ; 0.9\degree C ; 0.8\degree C ; 0.7\degree C ; 0.8\degree C ; 1\degree C ; 1.2\degree C. ; 1.4\degree C ; 1.6\degree C$$


\subsection*{Période 4 - 1987 - 2017}

$$Données : 1.6\degree C ; 1.3\degree C ; 1.5\degree C ; 1.8\degree C ; 2\degree C ; 1.7\degree C ; 2.2\degree C ; 2.4\degree C ; 2.8\degree C$$


\end{document}  
