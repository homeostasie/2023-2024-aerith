\documentclass[11pt]{article}
\usepackage{geometry,marginnote} % Pour passer au format A4
\geometry{hmargin=1cm, vmargin=1cm} % 

% Page et encodage
\usepackage[T1]{fontenc} % Use 8-bit encoding that has 256 glyphs
\usepackage[english,french]{babel} % Français et anglais
\usepackage[utf8]{inputenc} 

\usepackage{lmodern,numprint}
\setlength\parindent{0pt}

% Graphiques
\usepackage{graphicx,float,grffile,units}
\usepackage{tikz,pst-eucl,pst-plot,pstricks,pst-node,pstricks-add,pst-fun,pgfplots} 

% Maths et divers
\usepackage{amsmath,amsfonts,amssymb,amsthm,verbatim}
\usepackage{multicol,enumitem,url,eurosym,gensymb,tabularx}

\DeclareUnicodeCharacter{20AC}{\euro}



% Sections
\usepackage{sectsty} % Allows customizing section commands
\allsectionsfont{\centering \normalfont\scshape}

% Tête et pied de page
\usepackage{fancyhdr} \pagestyle{fancyplain} \fancyhead{} \fancyfoot{}

\renewcommand{\headrulewidth}{0pt} % Remove header underlines
\renewcommand{\footrulewidth}{0pt} % Remove footer underlines

\newcommand{\horrule}[1]{\rule{\linewidth}{#1}} % Create horizontal rule command with 1 argument of height

\newcommand{\Pointilles}[1][3]{%
  \multido{}{#1}{\makebox[\linewidth]{\dotfill}\\[\parskip]
}}

\newtheorem{Definition}{Définition}

\usepackage{siunitx}
\sisetup{
    detect-all,
    output-decimal-marker={,},
    group-minimum-digits = 3,
    group-separator={~},
    number-unit-separator={~},
    inter-unit-product={~}
}

\setlength{\columnseprule}{1pt}

\begin{document}

Chaque période dure 30ans. Pour chacune des périodes, on a une donnée de température tous les 4ans.
Pour chaque période, représenter les données avec un diagramme en bâtons.

On impose l'échelle : 
\begin{itemize}
      \item ordonnée : graduée de $0 \degree C$ à $3 \degree C$ avec 8cm pour $1 \degree C$.
      \item abscisse : graduée avec $1cm = 2ans$.
\end{itemize}

\subsection*{Période 1 - 1897 - 1927}



$$Données : 1\degree C ; 0.8\degree C ; 0.6\degree C ; 0.7\degree C ;  0.6\degree C ; 0.7\degree C ; 1\degree C ; 1.1\degree C ; 0.9\degree C$$


\subsection*{Période 2 - 1927 - 1957}

$$Données : 0.9\degree C ; 1.1\degree C ; 1.2\degree C ; 1.3\degree C ;  1.4\degree C ; 1.4\degree C ; 1.2\degree C ; 1.1\degree C ; 1.\degree C$$



\subsection*{Période 3 - 1957 - 1987}

$$Données : 1\degree C ; 0.9\degree C ; 0.8\degree C ; 0.7\degree C ; 0.8\degree C ; 1\degree C ; 1.2\degree C. ; 1.4\degree C ; 1.6\degree C$$


\subsection*{Période 4 - 1987 - 2017}

$$Données : 1.6\degree C ; 1.3\degree C ; 1.5\degree C ; 1.8\degree C ; 2\degree C ; 1.7\degree C ; 2.2\degree C ; 2.4\degree C ; 2.8\degree C$$


\end{document}  
