\input{../doc-class-cours.tex}

\begin{document}

\textbf{Nom, Prénom :} \hspace{8cm} \textbf{Classe :} \hspace{3cm} \textbf{Date :}\\

\vspace{-0.8cm}

\begin{center}
  \textit{Le plus grand ennemi de la connaissance n'est pas l'ignorance. C'est l'illusion de la connaissance.} 
  
  \textbf{Stephen Hawking}
\end{center}


\textbf{Ex 1 : Mettre sous forme scientifique}

\begin{multicols}{2}
  \begin{itemize}[label={$\bullet$}]
  \item $\SI{400000}{} = \dotfill$
  \item $\SI{342000000}{} = \dotfill$
  \item $\SI{-23400000}{} = \dotfill$
  \item $\SI{-14500000}{} = \dotfill$
  \item $\SI{0,00004}{} = \dotfill$
  \item $\SI{0,00000656}{} = \dotfill$
  \item $\SI{-0,000000000204}{} = \dotfill$
  \item $\SI{-0,0000235}{} = \dotfill$
  \end{itemize}
\end{multicols}


\textbf{Ex 2 : Démontrer que $10^4 \times 10^2 = 10^6$}

\Pointilles[3]


\textbf{pb1 :} La masse de la terre est $M_T = 5,9 \times 10^{24} kg$. La masse d'un trou noir est $\SI{30000000}{}$ fois plus lourde. 

\textbf{Quelle est la masse d'un trou noir ?}

\Pointilles[3]

\textbf{pb2 :} La lumière se déplace à la vitesse de $3 \times 10^8$ m/s. \textbf{Quelle distance parcourt-elle en 10 jours ?}

\Pointilles[3]

\textbf{pb3 :} Il y a $2^{30}$ galaxies dans notre univers. Chaque galaxies contient $3^{31}$ étoiles.  \\
Le volume de sable sur Terre est $1\,200 \text{ milliards de } m^3$. Chaque $m^3$ contient $510 \text{ milliards}$ de grains de sable. 

\textbf{John affirme qu'il y a plus de grains de sable sur Terre que d'étoiles dans l'univers ? A-t-il raison ?}

\Pointilles[5]

\textbf{pb4 :} Je possède un sac de 26 millions d’euros en billet de 20 \euro{}. Les billets de banque ont une épaisseur de $60 \times 10^{-6} m$.

\textbf{Quelle hauteur atteindrait la pile de billets ?}

\Pointilles[6]

\textbf{pb5 :} Une molécule de dioxyde de carbone est composée d'un atome de carbone (C) et de deux atomes d'oxygène (O) : $CO_2$. La masse d'un atome de carbone est $ m_c = 2 \times 10^{-26}kg$ et la masse d'un atome d'oxygène est $ m_O = 1,8 \times 10^{-26}kg$. 

\textbf{Combien trouve-t-on de molécules de dioxyde carbone dans 4 kg ?}

\Pointilles[6]

\newpage


\textbf{Nom, Prénom :} \hspace{8cm} \textbf{Classe :} \hspace{3cm} \textbf{Date :}\\

\vspace{-0.8cm}

\begin{center}
  \textit{Le plus grand ennemi de la connaissance n'est pas l'ignorance. C'est l'illusion de la connaissance.} 
  
  \textbf{Stephen Hawking}
\end{center}


\textbf{Ex 1 : Mettre sous forme scientifique}

\begin{multicols}{2}
  \begin{itemize}[label={$\bullet$}]
  \item $\SI{60000}{} = \dotfill$
  \item $\SI{63700000000}{} = \dotfill$
  \item $\SI{-12800}{} = \dotfill$
  \item $\SI{-2670000000}{} = \dotfill$
  \item $\SI{0,000006}{} = \dotfill$
  \item $\SI{0,0000307}{} = \dotfill$
  \item $\SI{-0,00000000544}{} = \dotfill$
  \item $\SI{-0,00000163}{} = \dotfill$
  \end{itemize}
\end{multicols}


\textbf{Ex 2 : Démontrer que $10^3 \times 10^4 = 10^7$}

\Pointilles[3]


\textbf{pb1 :} La masse de la terre est $M_T = 6,1 \times 10^{27} g$. La masse d'un trou noir est $\SI{40000000}{}$ fois plus lourde. 

\textbf{Quelle est la masse d'un trou noir ?}

\Pointilles[3]

\textbf{pb2 :} La lumière se déplace à la vitesse de $3 \times 10^8$ m/s. \textbf{Quelle distance parcourt-elle en 15 jours ?}

\Pointilles[3]

\textbf{pb3 :} Il y a $2^{31}$ galaxies dans notre univers. Chaque galaxies contient $3^{30}$ étoiles.  \\
Le volume de sable sur Terre est $1\,100 \text{ milliards de } m^3$. Chaque $m^3$ contient $620 \text{ milliards}$ de grains de sable. 

\textbf{John affirme qu'il y a plus de grains de sable sur Terre que d'étoiles dans l'univers ? A-t-il raison ?}

\Pointilles[5]

\textbf{pb4 :} Je possède un sac de 46 milliards d’euros en billet de 50 \euro{}. Les billets de banque ont une épaisseur de $80 \times 10^{-6} m$.

\textbf{Quelle hauteur atteindrait la pile de billets ?}

\Pointilles[6]

\textbf{pb5 :} Une molécule de dioxyde de carbone est composée d'un atome de carbone (C) et de deux atomes d'oxygène (O) : $CO_2$. La masse d'un atome de carbone est $ m_c = 3 \times 10^{-26}kg$ et la masse d'un atome d'oxygène est $ m_O = 1,9 \times 10^{-26}kg$. 

\textbf{Combien trouve-t-on de molécules de dioxyde carbone dans 6 kg ?}

\Pointilles[6]

\end{document}
