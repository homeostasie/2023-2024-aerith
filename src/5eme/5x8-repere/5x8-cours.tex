\input{../doc-class-cours.tex}

\begin{document}

\section*{8 - Repères}

\subsection*{1 - Vocabulaire}

\begin{Definition}{Un repère}\\
  Un repère permet d’identifier des points par des coordonnées
 \end{Definition}

 \textbf{Coordonnées : }

 \begin{itemize}[label={$\bullet$}]
   \item La première coordonnée s'appelle l'\textbf{abscisse}. Elle représente le "gauche / droite".
   \item La première coordonnée s'appelle l'\textbf{ordonnée}. Elle représente le "bas / haut".
 \end{itemize}

 \begin{Definition}{Un repère mathématique}\\
    \begin{itemize}[label={$\bullet$}]
    \item Les deux axes sont perpendiculaires.
    \item Les graduations sont régulières. 
  \end{itemize}
 \end{Definition}

 \subsection*{2 - Les exercices types}

1. Donner les coordonnées d'un point.
2. Placer les points dans le repère. 
3. Tracer un courbe à partir d'un tableau. 

\end{document}

