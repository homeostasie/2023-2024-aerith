\documentclass[11pt]{article}
\usepackage{geometry,marginnote} % Pour passer au format A4
\geometry{hmargin=1cm, vmargin=1cm} % 

% Page et encodage
\usepackage[T1]{fontenc} % Use 8-bit encoding that has 256 glyphs
\usepackage[english,french]{babel} % Français et anglais
\usepackage[utf8]{inputenc} 

\usepackage{lmodern,numprint}
\setlength\parindent{0pt}

% Graphiques
\usepackage{graphicx,float,grffile,units}
\usepackage{tikz,pst-eucl,pst-plot,pstricks,pst-node,pstricks-add,pst-fun,pgfplots} 

% Maths et divers
\usepackage{amsmath,amsfonts,amssymb,amsthm,verbatim}
\usepackage{multicol,enumitem,url,eurosym,gensymb,tabularx}

\DeclareUnicodeCharacter{20AC}{\euro}



% Sections
\usepackage{sectsty} % Allows customizing section commands
\allsectionsfont{\centering \normalfont\scshape}

% Tête et pied de page
\usepackage{fancyhdr} \pagestyle{fancyplain} \fancyhead{} \fancyfoot{}

\renewcommand{\headrulewidth}{0pt} % Remove header underlines
\renewcommand{\footrulewidth}{0pt} % Remove footer underlines

\newcommand{\horrule}[1]{\rule{\linewidth}{#1}} % Create horizontal rule command with 1 argument of height

\newcommand{\Pointilles}[1][3]{%
  \multido{}{#1}{\makebox[\linewidth]{\dotfill}\\[\parskip]
}}

\newtheorem{Definition}{Définition}

\usepackage{siunitx}
\sisetup{
    detect-all,
    output-decimal-marker={,},
    group-minimum-digits = 3,
    group-separator={~},
    number-unit-separator={~},
    inter-unit-product={~}
}

\setlength{\columnseprule}{1pt}

\begin{document}

\section*{8 - Repères}

\subsection*{1 - Vocabulaire}

\begin{Definition}{Un repère}\\
  Un repère permet d’identifier des points par des coordonnées
 \end{Definition}

 \textbf{Coordonnées : }

 \begin{itemize}[label={$\bullet$}]
   \item La première coordonnée s'appelle l'\textbf{abscisse}. Elle représente le "gauche / droite".
   \item La première coordonnée s'appelle l'\textbf{ordonnée}. Elle représente le "bas / haut".
 \end{itemize}

 \begin{Definition}{Un repère mathématique}\\
    \begin{itemize}[label={$\bullet$}]
    \item Les deux axes sont perpendiculaires.
    \item Les graduations sont régulières. 
  \end{itemize}
 \end{Definition}

 \subsection*{2 - Les exercices types}

1. Donner les coordonnées d'un point.
2. Placer les points dans le repère. 
3. Tracer un courbe à partir d'un tableau. 

\end{document}

