\input{../doc-class-cours.tex}

\begin{document}

\section*{4 - Symétrie centrale 1 : quadrillage}

\begin{Definition}{Symétrique d'un point}\\
  $A$ est le symétrique de $B$ par rapport à $O$ si $O$ est le milieu de $[AB]$.
\end{Definition}

\subsection*{1 - Le travail sur quadrillage}

\textbf{Les règles : }

\begin{itemize}[label={$\bullet$}]
  \item On place les points importants aux intersections
  \item On se déplace au point suivant en suivant un \textbf{motif}
\end{itemize} 

\textbf{Exemples :}

\begin{figure}[H]
  \centering
  \includegraphics[width=0.8\linewidth]{5x4-symetrie-centrale-1-quadrillage/motif.pdf}
\end{figure}

\begin{Definition}{Milieu d'un segment}\\
  $O$ est le milieu de $[AB]$ si $[AO]$ et $[OB]$ ont le même motif.
\end{Definition}

\textbf{Technique : }Pour tracer le symétrique d'un point sur quadrillage : \textbf{on reproduit le motif}.
\end{document}
