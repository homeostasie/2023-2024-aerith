\documentclass[11pt]{article}
\usepackage{geometry,marginnote} % Pour passer au format A4
\geometry{hmargin=1cm, vmargin=1cm} % 

% Page et encodage
\usepackage[T1]{fontenc} % Use 8-bit encoding that has 256 glyphs
\usepackage[english,french]{babel} % Français et anglais
\usepackage[utf8]{inputenc} 

\usepackage{lmodern,numprint}
\setlength\parindent{0pt}

% Graphiques
\usepackage{graphicx,float,grffile,units}
\usepackage{tikz,pst-eucl,pst-plot,pstricks,pst-node,pstricks-add,pst-fun,pgfplots} 

% Maths et divers
\usepackage{amsmath,amsfonts,amssymb,amsthm,verbatim}
\usepackage{multicol,enumitem,url,eurosym,gensymb,tabularx}

\DeclareUnicodeCharacter{20AC}{\euro}



% Sections
\usepackage{sectsty} % Allows customizing section commands
\allsectionsfont{\centering \normalfont\scshape}

% Tête et pied de page
\usepackage{fancyhdr} \pagestyle{fancyplain} \fancyhead{} \fancyfoot{}

\renewcommand{\headrulewidth}{0pt} % Remove header underlines
\renewcommand{\footrulewidth}{0pt} % Remove footer underlines

\newcommand{\horrule}[1]{\rule{\linewidth}{#1}} % Create horizontal rule command with 1 argument of height

\newcommand{\Pointilles}[1][3]{%
  \multido{}{#1}{\makebox[\linewidth]{\dotfill}\\[\parskip]
}}

\newtheorem{Definition}{Définition}

\usepackage{siunitx}
\sisetup{
    detect-all,
    output-decimal-marker={,},
    group-minimum-digits = 3,
    group-separator={~},
    number-unit-separator={~},
    inter-unit-product={~}
}

\setlength{\columnseprule}{1pt}

\begin{document}

\section*{4 - Symétrie centrale 1 : quadrillage}

\begin{Definition}{Symétrique d'un point}\\
  $A$ est le symétrique de $B$ par rapport à $O$ si $O$ est le milieu de $[AB]$.
\end{Definition}

\subsection*{1 - Le travail sur quadrillage}

\textbf{Les règles : }

\begin{itemize}[label={$\bullet$}]
  \item On place les points importants aux intersections
  \item On se déplace au point suivant en suivant un \textbf{motif}
\end{itemize} 

\textbf{Exemples :}

\begin{figure}[H]
  \centering
  \includegraphics[width=0.8\linewidth]{5x4-symetrie-centrale-1-quadrillage/motif.pdf}
\end{figure}

\begin{Definition}{Milieu d'un segment}\\
  $O$ est le milieu de $[AB]$ si $[AO]$ et $[OB]$ ont le même motif.
\end{Definition}

\textbf{Technique : }Pour tracer le symétrique d'un point sur quadrillage : \textbf{on reproduit le motif}.
\end{document}
