\input{../doc-class-cours.tex}

\begin{document}

\section*{1 - Calcul Numérique}

\subsection*{1 - L'usage de la calculatrice}

\begin{itemize}[label={$\bullet$}]
  \item La calculatrice est obligatoire à chaque séance.
  \item La calculatrice est autorisée lors des évaluations mais personnelle.
  \item La calculatrice ne dispense pas d'écrire les calculs.
  \item Pas de calculatrice = votre problème.
\end{itemize}

\subsection*{2 - La rédaction des calculs}

\begin{itemize}[label={$\bullet$}]
  \item Tout ce qui est tapé à la calculatrice doit être écrit. 
  \item Dans un problème : On écrit le calcul qu'on veut faire.
  \item Dans un exercice : On fait des étapes.
\end{itemize}

\begin{Definition}{Sens du signe =}\\
  Ce qui est à gauche = Ce qui est à droite
\end{Definition}

\textbf{Remarques : }
\begin{itemize}[label={$\bullet$}]
  \item Un seul égal par ligne. On essaye de les aligner. Le signe = ne signifie pas : "Le résultat de ".
  \item On peut souligner les opérations à faire en priorité. 
\end{itemize}

\subsection*{3 - Les priorités de calculs}

Un calcul se fait généralement dans le sens de la lecture... mais \textbf{pas toujours}.

\begin{enumerate}
  \item[1.] Les multiplications et les divisions sont prioritaires par rapport aux additions et soustractions.
  \item[2.] Les parenthèses signalent que le calcul dedans doit être fait en premier.  
\end{enumerate}

\textbf{Pour la rédaction :}

\begin{itemize}[label={$\bullet$}]
  \item On lit le calcul en entier et on repère l'opération à faire en priorité.
  \item On recopie tout le calcul en remplaçant l'opération par son résultat.
  \item On peut faire plusieurs opérations en même temps si elles sont indépendantes. 
\end{itemize}

\subsection*{4 - Les exercices}

\textbf{Ex1 : Calculer}

\begin{multicols}{3}\begin{itemize}[label={$\bullet$}]
  \item $A = 5 \times (3 + 7) $ \\
        $A = 5 \times 10 $ \\
        $A = 50 $ 

  \item $B = 2 + 8 \times 5 $ \\
        $B = 2 + 40$ \\
        $B = 42$ 

  \item $C = 6 \times 4 \div 3 + 2 \times 5 $\\
        $C = 24 \div 3 + 10 $\\
        $C = 8 + 10 $\\
        $C = 18 $
\end{itemize}\end{multicols}

\textbf{Ex2 : Problème}

Enès achète un compas à 2,20€ et 4 stylos manga à 1,30€ l'unité. Combien paie-t-il ? 

\begin{flalign*}
  2,20 + 4 \times 1,30 &= 2,20 + 5,20 &\\
                       &= 7,40
\end{flalign*}

Enès paie 7,40€. \\

\end{document}
