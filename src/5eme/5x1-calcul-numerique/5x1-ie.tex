\documentclass[11pt]{article}
\usepackage{geometry,marginnote} % Pour passer au format A4
\geometry{hmargin=1cm, vmargin=1cm} % 

% Page et encodage
\usepackage[T1]{fontenc} % Use 8-bit encoding that has 256 glyphs
\usepackage[english,french]{babel} % Français et anglais
\usepackage[utf8]{inputenc} 

\usepackage{lmodern,numprint}
\setlength\parindent{0pt}

% Graphiques
\usepackage{graphicx,float,grffile,units}
\usepackage{tikz,pst-eucl,pst-plot,pstricks,pst-node,pstricks-add,pst-fun,pgfplots} 

% Maths et divers
\usepackage{amsmath,amsfonts,amssymb,amsthm,verbatim}
\usepackage{multicol,enumitem,url,eurosym,gensymb,tabularx}

\DeclareUnicodeCharacter{20AC}{\euro}



% Sections
\usepackage{sectsty} % Allows customizing section commands
\allsectionsfont{\centering \normalfont\scshape}

% Tête et pied de page
\usepackage{fancyhdr} \pagestyle{fancyplain} \fancyhead{} \fancyfoot{}

\renewcommand{\headrulewidth}{0pt} % Remove header underlines
\renewcommand{\footrulewidth}{0pt} % Remove footer underlines

\newcommand{\horrule}[1]{\rule{\linewidth}{#1}} % Create horizontal rule command with 1 argument of height

\newcommand{\Pointilles}[1][3]{%
  \multido{}{#1}{\makebox[\linewidth]{\dotfill}\\[\parskip]
}}

\newtheorem{Definition}{Définition}

\usepackage{siunitx}
\sisetup{
    detect-all,
    output-decimal-marker={,},
    group-minimum-digits = 3,
    group-separator={~},
    number-unit-separator={~},
    inter-unit-product={~}
}

\setlength{\columnseprule}{1pt}

\begin{document}

\textbf{Nom, Prénom :} \hspace{8cm} \textbf{Classe :} \hspace{3cm} \textbf{Date :}\\

\begin{center}
  \textit{Le plus grand ennemi de la connaissance n'est pas l'ignorance. C'est l'illusion de la connaissance.} 
  
  \textbf{Stephen Hawking}
\end{center}

\textbf{Définition: Sens du signe = :} \\ \Pointilles[1]

\textbf{Ex1 : Calculer}

\begin{multicols}{3}\begin{itemize}[label={$\bullet$}]
  \item $A = 1 + 2 \times 3$ 
  \item $B = (2+3) \times (5+15)$
  \item $C = 10 \times 2 \div 5 + 1$
  \item $D = (10 - (7 + 1)) \times (3 + 2)$
  \item $E = 12,1 + 2,4 - 3,6 + 2,1 \times 4.2$
  \item $F = 3 + 12 \div 4 + 1 \times (1 + 2 \times 4)$
\end{itemize}\end{multicols}

\Pointilles[20]

\textbf{Problèmes : Écrire l'expression, calculer avec les étapes, faire une phrase réponse.}

\begin{multicols}{2}

\textbf{pb1 : }
Sofia achète un livre à 4€ et 5 manga à 6€ l'unité. Combien paie-t-elle ?  \\ \Pointilles[5]

\textbf{pb2 : }
Un camion pèse 2500 kg. Enzo décharge 3 caisses de 140kg chacune. Combien pèse le camion ? \\ \Pointilles[5] \columnbreak

\textbf{pb3 : }
\og J'ai multiplié 2,39 par 10 puis j'ai ajouté le produit de 45 par 0,1. \fg 
Quel est le résultat ? \\ \Pointilles[5]

\textbf{pb4 : }
Olivia prépare 5 bouquets qui auront chacun 5 roses blanches et 8 roses rouges. Combien lui faut-il de roses en tout ? \\ \Pointilles[5]

\end{multicols}

\end{document}
