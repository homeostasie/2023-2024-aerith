\input{../doc-class-cours.tex}

\begin{document}

\textbf{Nom, Prénom :} \hspace{8cm} \textbf{Classe :} \hspace{3cm} \textbf{Date :}\\

\begin{center}
  \textit{L’essentiel est sans cesse menacé par l’insignifiant.} - \textbf{René Char}
\end{center}

\textbf{Définition: Probabilité} \\ \Pointilles[2]

\textbf{Ex1 - Pièces}

Chaque pièce n'a que 2 côtés : Pile (\textbf{P}) et Face (\textbf{F}).

\begin{enumerate}
  \item[1a.] On jette 2 pièces. Écrire toutes les possibilités. \\\Pointilles[1]
  \item[1b.] On jette 3 pièces. Écrire toutes les possibilités. \\\Pointilles[3]
  \item[1c.] Je viens de faire 3 fois Pile en jetant ma pièce. Je la jette à nouveau. Quelle est la probabilité de faire à nouveau Pile ? \dotfill
\end{enumerate}  

\textbf{Ex2 - Dé}

On a un dé équilibré à 6 faces : 1, 1, 2, 2, 3, 4.

\begin{enumerate}
  \item[2a.] Quel est l'univers ? \dotfill
  \item[2b.] Quels sont les probabilités associées à chaque issue ? \\ \Pointilles[2]
\end{enumerate}  


\textbf{Ex3 - Dés}

\begin{multicols}{2}\noindent
  On a deux dés à 6 faces numérotés 1, 2, 3, 4, 5 et 6.
   
  \textbf{On additionne les deux faces.} 
  \begin{enumerate}
    \item[3a.] Remplir le tableau des réalisations.
    \item[3b.] Décrire l'univers ? 
  \end{enumerate}  
  \columnbreak 
  
  \begin{center}\begin{tabular}{|c|c|c|c|c|c|c|} \hline
    dés & 1 & 2 & 3 & 4 & 5 & 6 \\  \hline
      1 &   &   &   &   &   &   \\  \hline
      2 &   &   &   &   &   &   \\  \hline
      3 &   &   &   &   &   &   \\  \hline
      4 &   &   &   &   &   &   \\  \hline
      5 &   &   &   &   &   &   \\  \hline
      6 &   &   &   &   &   &   \\  \hline
  \end{tabular}\end{center}
  
  \end{multicols}
  
  \Pointilles[2] 
  \begin{enumerate}
    \item[3c.] Quelle est la probabilité de faire 6 ? \\ \Pointilles[2] 
    \item[3d.] Quelle nombre à le plus de chance de sortir ? \\ \Pointilles[2]  
  \end{enumerate}  

\end{document}
