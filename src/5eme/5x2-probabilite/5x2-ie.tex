\documentclass[11pt]{article}
\usepackage{geometry,marginnote} % Pour passer au format A4
\geometry{hmargin=1cm, vmargin=1cm} % 

% Page et encodage
\usepackage[T1]{fontenc} % Use 8-bit encoding that has 256 glyphs
\usepackage[english,french]{babel} % Français et anglais
\usepackage[utf8]{inputenc} 

\usepackage{lmodern,numprint}
\setlength\parindent{0pt}

% Graphiques
\usepackage{graphicx,float,grffile,units}
\usepackage{tikz,pst-eucl,pst-plot,pstricks,pst-node,pstricks-add,pst-fun,pgfplots} 

% Maths et divers
\usepackage{amsmath,amsfonts,amssymb,amsthm,verbatim}
\usepackage{multicol,enumitem,url,eurosym,gensymb,tabularx}

\DeclareUnicodeCharacter{20AC}{\euro}



% Sections
\usepackage{sectsty} % Allows customizing section commands
\allsectionsfont{\centering \normalfont\scshape}

% Tête et pied de page
\usepackage{fancyhdr} \pagestyle{fancyplain} \fancyhead{} \fancyfoot{}

\renewcommand{\headrulewidth}{0pt} % Remove header underlines
\renewcommand{\footrulewidth}{0pt} % Remove footer underlines

\newcommand{\horrule}[1]{\rule{\linewidth}{#1}} % Create horizontal rule command with 1 argument of height

\newcommand{\Pointilles}[1][3]{%
  \multido{}{#1}{\makebox[\linewidth]{\dotfill}\\[\parskip]
}}

\newtheorem{Definition}{Définition}

\usepackage{siunitx}
\sisetup{
    detect-all,
    output-decimal-marker={,},
    group-minimum-digits = 3,
    group-separator={~},
    number-unit-separator={~},
    inter-unit-product={~}
}

\setlength{\columnseprule}{1pt}

\begin{document}

\textbf{Nom, Prénom :} \hspace{8cm} \textbf{Classe :} \hspace{3cm} \textbf{Date :}\\

\begin{center}
  \textit{L’essentiel est sans cesse menacé par l’insignifiant.} - \textbf{René Char}
\end{center}

\textbf{Définition: Probabilité} \\ \Pointilles[4]

\textbf{Ex1 - Pièces}

Chaque pièce n'a que 2 côtés : Pile (\textbf{P}) et Face (\textbf{F}).

\begin{enumerate}
  \item[1.] On jette 2 pièces. Quels sont tous les cas possibles ? \\\Pointilles[2]
  \item[2a.] On jette 3 pièces. Quels sont tous les cas possibles ? \\\Pointilles[3]
  \item[2b.] Quelle est la probabilité de faire FFF ?  \\\Pointilles[2]
  \item[2c.] Je viens de faire 2 fois Face en jetant les deux premières pièces. Je jette une pièce à nouveau. Quelle est la probabilité de faire à nouveau Face ? \\\Pointilles[1]
  \item[B.] Je jette 12 pièces. Quelle est la taille de l'univers ? \\\Pointilles[2]
\end{enumerate}  

\textbf{Ex2 - Dé}

On a un dé équilibré à 8 faces : 1, 1, 1, 2, 3, 3, 3, 4.

\begin{enumerate}
  \item[1.] Quels sont tous les cas possibles ? \\ \Pointilles[2]
  \item[2.] Quelles sont les probabilités associées à chaque issue ? \\ \Pointilles[2]
\end{enumerate}  

\newpage

\textbf{pb1 - 2 Dés}

\begin{multicols}{2}\noindent
  On a deux dés à 6 faces numérotés : 
  \begin{itemize}[label={$\bullet$}]
    \item 1, 2, 2, 4, 4 et 5
    \item 1, 1, 3, 3, 5 et 6
  \end{itemize}

  \textbf{On conserve uniquement le dé le plus petit.} 
  \begin{enumerate}
    \item[1.] Remplir le tableau des réalisations.
    \item[2.] Quels sont tous les cas possibles ? 
  \end{enumerate}  
  \columnbreak 
  
  \begin{center}\begin{tabular}{|c|c|c|c|c|c|c|} \hline
    dés & 1 & 2 & 2 & 4 & 4 & 5 \\  \hline
      1 &   &   &   &   &   &   \\  \hline
      1 &   &   &   &   &   &   \\  \hline
      3 &   &   &   &   &   &   \\  \hline
      3 &   &   &   &   &   &   \\  \hline
      5 &   &   &   &   &   &   \\  \hline
      6 &   &   &   &   &   &   \\  \hline
  \end{tabular}\end{center}
  
  \end{multicols}
  
  \Pointilles[3] 
  \begin{enumerate}
    \item[3.] Quelles sont les probabilités associées à chaque issue ? \\ \Pointilles[6]
    \item[4.] Quelle est la probabilité d'obtenir plus grand ou égal à 3 ?\\ \Pointilles[3]  
  \end{enumerate}  


  \textbf{pb2 - Dé et Urnes}

  On a un dé équilibrés à 6 faces : 1, 1, 6, 6, 6, 6. On lance le dé. 
  
  \begin{multicols}{2}\noindent
  \begin{itemize}[label={$\bullet$}]
    \item Si on obtient 1 avec le dé, on pioche une balle dans l'urne rouge.
    Urne rouge : 14 balles rouges et 10 balles noires.
  
    \item Si on obtient 6 avec le dé, on pioche une balle dans l'urne verte.
    Urne verte : 8 balles vertes et 22 balles noires.
  \end{itemize}
  \end{multicols}
  
  \begin{enumerate}
    \item[1.] On fait 1 avec le dé. Quelle est la probabilité de piocher une balle noire ? \\ \Pointilles[3] 
    \item[2.] Quel est la probabilité de piocher dans l'urne rouge ? \\ \Pointilles[3] 
    \item[3.] On sait que la probabilité de piocher une balle noire est $\dfrac{2}{6} \times \dfrac{10}{24} + \dfrac{4}{6} \times \dfrac{22}{30}$. \\
    Calculer cette probabilité. A-t-on plus d'une chance sur deux de tirer une balle noire ? \\ \Pointilles[3] 
  \end{enumerate}
  
\end{document}
