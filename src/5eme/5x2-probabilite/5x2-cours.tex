\input{../doc-class-cours.tex}

\begin{document}

\section*{2 - Probabilité}

\subsection*{1 - Le vocabulaire}

\begin{Definition}{Hasard mathématique}
  \begin{enumerate}
    \item[1.] On sait ce qui peut se passer.
    \item[2.] On ne sait pas ce qui va se passer.
    \item[3.] L'expérience est reproductible.
  \end{enumerate}
\end{Definition}

\textbf{Exemple :} Je jette un dé classique à 6 faces. 
Je peux faire 1,2,3,4,5,6 ; mais je ne sais pas ce que je vais faire ; je peux rejouer autant que je veux. \\

\textbf{Remarque : } \textit{Le chapitre de probabilité est surtout un chapitre mental.} \\


\textbf{Vocabulaire : } 
\begin{itemize}[label={$\bullet$}]
  \item Une issue est une possibilité. 
  \item L'ensemble des issues est l'univers.
  \item Une réalisation est une "partie". 
\end{itemize}

\begin{Definition}{Probabilité}\\
  La probabilité est un nombre entre 0 et 1 qui évalue la chance qu'une issue se réalise. 
\end{Definition}

\subsection*{2 - Les pièces}

Les pièces ont deux côtés : Pile (P) et Face (F).
\begin{itemize}[label={$\bullet$}]
  \item La probabilité de faire Pile ou Face est $\dfrac{1}{2} = 0.5$.
  \item Faire Pile au premier lancer n'implique pas de faire Face au lancer suivant. La probabilité reste de $\dfrac{1}{2}$ pour Pile ou Face.
  \item On peut jeter plusieurs pièces en même temps.
  \item Il faut dénombrer. 
\end{itemize}

\subsection*{3 - Les dés}

Les dés ont souvent 6 faces : 1,2,3,4,5,6.

\begin{itemize}[label={$\bullet$}]
  \item L'univers est 1,2,3,4,5,6
  \item La probabilité de faire 4 est $\dfrac{1}{6}$.
  \item Il est très facile d'imaginer des dés différents.
  \item Il est très facile d'associer des dés.
  \item Il faut faire un tableau.
\end{itemize}

\end{document}
