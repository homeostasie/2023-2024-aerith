\documentclass[11pt]{article}
\usepackage{geometry,marginnote} % Pour passer au format A4
\geometry{hmargin=1cm, vmargin=1cm} % 

% Page et encodage
\usepackage[T1]{fontenc} % Use 8-bit encoding that has 256 glyphs
\usepackage[english,french]{babel} % Français et anglais
\usepackage[utf8]{inputenc} 

\usepackage{lmodern,numprint}
\setlength\parindent{0pt}

% Graphiques
\usepackage{graphicx,float,grffile,units}
\usepackage{tikz,pst-eucl,pst-plot,pstricks,pst-node,pstricks-add,pst-fun,pgfplots} 

% Maths et divers
\usepackage{amsmath,amsfonts,amssymb,amsthm,verbatim}
\usepackage{multicol,enumitem,url,eurosym,gensymb,tabularx}

\DeclareUnicodeCharacter{20AC}{\euro}



% Sections
\usepackage{sectsty} % Allows customizing section commands
\allsectionsfont{\centering \normalfont\scshape}

% Tête et pied de page
\usepackage{fancyhdr} \pagestyle{fancyplain} \fancyhead{} \fancyfoot{}

\renewcommand{\headrulewidth}{0pt} % Remove header underlines
\renewcommand{\footrulewidth}{0pt} % Remove footer underlines

\newcommand{\horrule}[1]{\rule{\linewidth}{#1}} % Create horizontal rule command with 1 argument of height

\newcommand{\Pointilles}[1][3]{%
  \multido{}{#1}{\makebox[\linewidth]{\dotfill}\\[\parskip]
}}

\newtheorem{Definition}{Définition}

\usepackage{siunitx}
\sisetup{
    detect-all,
    output-decimal-marker={,},
    group-minimum-digits = 3,
    group-separator={~},
    number-unit-separator={~},
    inter-unit-product={~}
}

\setlength{\columnseprule}{1pt}

\begin{document}

\section*{5 - Fraction - 1 : Introduction}

\subsection*{1 - La définition}

\begin{Definition}{Une fraction}\\
  Une fraction est un nombre qui est défini par un calcul. On appelle ce nombre la valeur d'une fraction. \\
  $\dfrac{1}{3}$ est le nombre qui est solution du problème : $3 \times ... = 1$
\end{Definition}

\textbf{Exemples : }

\begin{itemize}[label={$\bullet$}]
  \item $\dfrac{5}{13}$ est le nombre qui est solution du problème : $13 \times ... = 5$
  \item $\dfrac{17}{7}$ est le nombre qui est solution du problème : $7 \times ... = 17$
\end{itemize}

\textbf{Remarques : }

\begin{itemize}[label={$\bullet$}]
  \item Le nombre du haut s'appelle le numérateur.
  \item Le nombre du bas s'appelle le dénominateur.
\end{itemize}

\subsection*{2 - Les fractions égales}


\begin{Definition}{Fractions égales}\\
  la valeur d'une fraction ne change pas quand on multiplie ou divise le numérateur et le dénominateur par un même nombre non nul. On parle de fractions égales.
\end{Definition}

\textbf{Exemple : }

\begin{itemize}[label={$\bullet$}]
  \item $\dfrac{1}{2} = \dfrac{2}{4} = \dfrac{3}{6} = \dfrac{5}{10} = \dfrac{40}{80} = \dfrac{500}{1000} = ... = 0,5$ 
\end{itemize}

\textbf{Remarque : }

\begin{itemize}[label={$\bullet$}]
  \item Pour chaque fraction, il existe une infinité de fractions égales.
\end{itemize}

\end{document}