\input{../doc-class-cours.tex}

\begin{document}

\section*{5 - Fraction - 1 : Introduction}

\subsection*{1 - La définition}

\begin{Definition}{Une fraction}\\
  Une fraction est un nombre qui est défini par un calcul. On appelle ce nombre la valeur d'une fraction. \\
  $\dfrac{1}{3}$ est le nombre qui est solution du problème : $3 \times ... = 1$
\end{Definition}

\textbf{Exemples : }

\begin{itemize}[label={$\bullet$}]
  \item $\dfrac{5}{13}$ est le nombre qui est solution du problème : $13 \times ... = 5$
  \item $\dfrac{17}{7}$ est le nombre qui est solution du problème : $7 \times ... = 17$
\end{itemize}

\textbf{Remarques : }

\begin{itemize}[label={$\bullet$}]
  \item Le nombre du haut s'appelle le numérateur.
  \item Le nombre du bas s'appelle le dénominateur.
\end{itemize}

\subsection*{2 - Les fractions égales}


\begin{Definition}{Fractions égales}\\
  la valeur d'une fraction ne change pas quand on multiplie ou divise le numérateur et le dénominateur par un même nombre non nul. On parle de fractions égales.
\end{Definition}

\textbf{Exemple : }

\begin{itemize}[label={$\bullet$}]
  \item $\dfrac{1}{2} = \dfrac{2}{4} = \dfrac{3}{6} = \dfrac{5}{10} = \dfrac{40}{80} = \dfrac{500}{1000} = ... = 0,5$ 
\end{itemize}

\textbf{Remarque : }

\begin{itemize}[label={$\bullet$}]
  \item Pour chaque fraction, il existe une infinité de fractions égales.
\end{itemize}

\end{document}