\documentclass[11pt]{article}
\usepackage{geometry,marginnote} % Pour passer au format A4
\geometry{hmargin=1cm, vmargin=1cm} % 

% Page et encodage
\usepackage[T1]{fontenc} % Use 8-bit encoding that has 256 glyphs
\usepackage[english,french]{babel} % Français et anglais
\usepackage[utf8]{inputenc} 

\usepackage{lmodern,numprint}
\setlength\parindent{0pt}

% Graphiques
\usepackage{graphicx,float,grffile,units}
\usepackage{tikz,pst-eucl,pst-plot,pstricks,pst-node,pstricks-add,pst-fun,pgfplots} 

% Maths et divers
\usepackage{amsmath,amsfonts,amssymb,amsthm,verbatim}
\usepackage{multicol,enumitem,url,eurosym,gensymb,tabularx}

\DeclareUnicodeCharacter{20AC}{\euro}



% Sections
\usepackage{sectsty} % Allows customizing section commands
\allsectionsfont{\centering \normalfont\scshape}

% Tête et pied de page
\usepackage{fancyhdr} \pagestyle{fancyplain} \fancyhead{} \fancyfoot{}

\renewcommand{\headrulewidth}{0pt} % Remove header underlines
\renewcommand{\footrulewidth}{0pt} % Remove footer underlines

\newcommand{\horrule}[1]{\rule{\linewidth}{#1}} % Create horizontal rule command with 1 argument of height

\newcommand{\Pointilles}[1][3]{%
  \multido{}{#1}{\makebox[\linewidth]{\dotfill}\\[\parskip]
}}

\newtheorem{Definition}{Définition}

\usepackage{siunitx}
\sisetup{
    detect-all,
    output-decimal-marker={,},
    group-minimum-digits = 3,
    group-separator={~},
    number-unit-separator={~},
    inter-unit-product={~}
}

\setlength{\columnseprule}{1pt}

\begin{document}

\section*{6 - Triangles - 1 : ILes angles}

\subsection*{1 - La définition}

\begin{Definition}{Un triangle}\\
 Un triangle est une figure géométrique formée par trois sommets distincts et par les trois segments qui les relient.
\end{Definition}

\textbf{Remarque : }

\begin{itemize}[label={$\bullet$}]
  \item En chaque sommet les côtés délimitent un angle intérieur. 
  \item Il y a donc trois angles : tri-angles.
\end{itemize}


\subsection*{Somme des angles}

\begin{Definition}{La propriété des angles}\\
  La somme des angles dans un triangle fait 180°.
\end{Definition}


\subsection*{Calculer des angles dans un triangle}

\textit{Il faut partir de la propriété puis chercher ce qu'on veut.} \\

\begin{minipage}[t]{0.3\textwidth}
  \begin{figure}[H]
    \centering
    \includegraphics[width=0.7\linewidth]{5x6-triangles-1-angles/pro-angle.pdf}
  \end{figure} 
\end{minipage}
\begin{minipage}[t]{0.33\textwidth}
  On peut rédiger :
  \begin{itemize}[label={$\bullet$}]
    \item 35 + 90 + ? = 180° 
    \item 35 + 90 = 125
    \item 180 - 125 = 55
    \item ? = 55°
  \end{itemize}  
\end{minipage}
\begin{minipage}[t]{0.33\textwidth}
  Ou bien :
  \begin{itemize}[label={$\bullet$}]
    \item 35 + 90 + ? = 180° 
    \item 180 - (35 + 90) = 55
    \item ? = 55°
  \end{itemize}  
\end{minipage}

\end{document}