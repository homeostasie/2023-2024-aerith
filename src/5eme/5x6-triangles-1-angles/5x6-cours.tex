\input{../doc-class-cours.tex}

\begin{document}

\section*{6 - Triangles - 1 : Les angles}

\subsection*{1 - La définition}

\begin{Definition}{Un triangle}\\
 Un triangle est une figure géométrique formée par trois sommets distincts et par les trois segments qui les relient.
\end{Definition}

\textbf{Remarque : }

\begin{itemize}[label={$\bullet$}]
  \item En chaque sommet les côtés délimitent un angle intérieur. 
  \item Il y a donc trois angles : tri-angles.
\end{itemize}


\subsection*{2 - Somme des angles}

\begin{Definition}{La propriété des angles}\\
  La somme des angles dans un triangle fait 180°.
\end{Definition}


\subsection*{3 - Calculer des angles dans un triangle}

\textit{Il faut partir de la propriété puis chercher ce qu'on veut.} \\

\begin{minipage}[t]{0.3\textwidth}
  \begin{figure}[H]
    \centering
    \includegraphics[width=0.7\linewidth]{5x6-triangles-1-angles/pro-angle.pdf}
  \end{figure} 
\end{minipage}
\begin{minipage}[t]{0.33\textwidth}
  On peut rédiger :
  \begin{itemize}[label={$\bullet$}]
    \item 35 + 90 + ? = 180° 
    \item 35 + 90 = 125
    \item 180 - 125 = 55
    \item ? = 55°
  \end{itemize}  
\end{minipage}
\begin{minipage}[t]{0.33\textwidth}
  Ou bien :
  \begin{itemize}[label={$\bullet$}]
    \item 35 + 90 + ? = 180° 
    \item 180 - (35 + 90) = 55
    \item ? = 55°
  \end{itemize}  
\end{minipage}

\end{document}