\documentclass[11pt]{article}
\usepackage{geometry,marginnote} % Pour passer au format A4
\geometry{hmargin=1cm, vmargin=1cm} % 

% Page et encodage
\usepackage[T1]{fontenc} % Use 8-bit encoding that has 256 glyphs
\usepackage[english,french]{babel} % Français et anglais
\usepackage[utf8]{inputenc} 

\usepackage{lmodern,numprint}
\setlength\parindent{0pt}

% Graphiques
\usepackage{graphicx,float,grffile,units}
\usepackage{tikz,pst-eucl,pst-plot,pstricks,pst-node,pstricks-add,pst-fun,pgfplots} 

% Maths et divers
\usepackage{amsmath,amsfonts,amssymb,amsthm,verbatim}
\usepackage{multicol,enumitem,url,eurosym,gensymb,tabularx}

\DeclareUnicodeCharacter{20AC}{\euro}



% Sections
\usepackage{sectsty} % Allows customizing section commands
\allsectionsfont{\centering \normalfont\scshape}

% Tête et pied de page
\usepackage{fancyhdr} \pagestyle{fancyplain} \fancyhead{} \fancyfoot{}

\renewcommand{\headrulewidth}{0pt} % Remove header underlines
\renewcommand{\footrulewidth}{0pt} % Remove footer underlines

\newcommand{\horrule}[1]{\rule{\linewidth}{#1}} % Create horizontal rule command with 1 argument of height

\newcommand{\Pointilles}[1][3]{%
  \multido{}{#1}{\makebox[\linewidth]{\dotfill}\\[\parskip]
}}

\newtheorem{Definition}{Définition}

\usepackage{siunitx}
\sisetup{
    detect-all,
    output-decimal-marker={,},
    group-minimum-digits = 3,
    group-separator={~},
    number-unit-separator={~},
    inter-unit-product={~}
}

\setlength{\columnseprule}{1pt}

\begin{document}

\section*{7 - Proportionnalité}

\subsection*{1 - La définition}

\begin{Definition}{Proportionnalité}\\
  Un tableau est proportionnel si un peu passer d'une ligne à l'autre en multipliant par un même nombre.
\end{Definition}

\begin{center} \begin{tabular}{|c|c|c|c|} \hline
  2 &  6 & 10 & 11   \\ \hline
  5 & 15 & 25 & 27.5 \\ \hline 
 \end{tabular}\end{center}

\textbf{Remarques : }
\begin{itemize}[label={$\bullet$}]
  \item Ce nombre s'appelle le coefficient de proportionnalité.
  \item Ce nombre est solution du problème : $2 \times ... = 5$. La solution est : $\dfrac{5}{2}$
\end{itemize}  

\subsection*{2 - Calculer}

\textbf{Méthode : }
\begin{itemize}[label={$\bullet$}]
  \item Pour calculer un nombre en haut, on multiplie par le coefficient de proportionnalité. 
  \item Pour calculer un nombre en bas, on divise par le coefficient de proportionnalité. 
  \item À la calculatrice, il faut utiliser la fraction plutôt que la valeur approchée. 
\end{itemize}  


\subsection*{3 - Vérifier si un tableau est proportionnel}

\begin{Definition}{Démonstration}\\
  Un tableau est proportionnel si le coefficient de chaque colonne est le même. 
\end{Definition}

\textbf{Méthode : }
\begin{itemize}[label={$\bullet$}]
  \item On écrit la fraction correspondante à chaque colonne.
  \item On calcule les fractions.
  \item Si elles sont égales : Le tableau est proportionnel.
  \item Si au moins deux coefficients sont différents. Le tableau n'est pas proportionnel.
\end{itemize}  

\end{document}