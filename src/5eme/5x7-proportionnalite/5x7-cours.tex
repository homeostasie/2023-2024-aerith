\input{../doc-class-cours.tex}

\begin{document}

\section*{7 - Proportionnalité}

\subsection*{1 - La définition}

\begin{Definition}{Proportionnalité}\\
  Un tableau est proportionnel si un peu passer d'une ligne à l'autre en multipliant par un même nombre.
\end{Definition}

\begin{center} \begin{tabular}{|c|c|c|c|} \hline
  2 &  6 & 10 & 11   \\ \hline
  5 & 15 & 25 & 27.5 \\ \hline 
 \end{tabular}\end{center}

\textbf{Remarques : }
\begin{itemize}[label={$\bullet$}]
  \item Ce nombre s'appelle le coefficient de proportionnalité.
  \item Ce nombre est solution du problème : $2 \times ... = 5$. La solution est : $\dfrac{5}{2}$
\end{itemize}  

\subsection*{2 - Calculer}

\textbf{Méthode : }
\begin{itemize}[label={$\bullet$}]
  \item Pour calculer un nombre en haut, on multiplie par le coefficient de proportionnalité. 
  \item Pour calculer un nombre en bas, on divise par le coefficient de proportionnalité. 
  \item À la calculatrice, il faut utiliser la fraction plutôt que la valeur approchée. 
\end{itemize}  


\subsection*{3 - Vérifier si un tableau est proportionnel}

\begin{Definition}{Démonstration}\\
  Un tableau est proportionnel si le coefficient de chaque colonne est le même. 
\end{Definition}

\textbf{Méthode : }
\begin{itemize}[label={$\bullet$}]
  \item On écrit la fraction correspondante à chaque colonne.
  \item On calcule les fractions.
  \item Si elles sont égales : Le tableau est proportionnel.
  \item Si au moins deux coefficients sont différents. Le tableau n'est pas proportionnel.
\end{itemize}  

\end{document}