\documentclass[11pt]{article}
\usepackage{geometry,marginnote} % Pour passer au format A4
\geometry{hmargin=1cm, vmargin=1cm} % 

% Page et encodage
\usepackage[T1]{fontenc} % Use 8-bit encoding that has 256 glyphs
\usepackage[english,french]{babel} % Français et anglais
\usepackage[utf8]{inputenc} 

\usepackage{lmodern,numprint}
\setlength\parindent{0pt}

% Graphiques
\usepackage{graphicx,float,grffile,units}
\usepackage{tikz,pst-eucl,pst-plot,pstricks,pst-node,pstricks-add,pst-fun,pgfplots} 

% Maths et divers
\usepackage{amsmath,amsfonts,amssymb,amsthm,verbatim}
\usepackage{multicol,enumitem,url,eurosym,gensymb,tabularx}

\DeclareUnicodeCharacter{20AC}{\euro}



% Sections
\usepackage{sectsty} % Allows customizing section commands
\allsectionsfont{\centering \normalfont\scshape}

% Tête et pied de page
\usepackage{fancyhdr} \pagestyle{fancyplain} \fancyhead{} \fancyfoot{}

\renewcommand{\headrulewidth}{0pt} % Remove header underlines
\renewcommand{\footrulewidth}{0pt} % Remove footer underlines

\newcommand{\horrule}[1]{\rule{\linewidth}{#1}} % Create horizontal rule command with 1 argument of height

\newcommand{\Pointilles}[1][3]{%
  \multido{}{#1}{\makebox[\linewidth]{\dotfill}\\[\parskip]
}}

\newtheorem{Definition}{Définition}

\usepackage{siunitx}
\sisetup{
    detect-all,
    output-decimal-marker={,},
    group-minimum-digits = 3,
    group-separator={~},
    number-unit-separator={~},
    inter-unit-product={~}
}

\setlength{\columnseprule}{1pt}

\begin{document}

\textbf{Nom, Prénom :} \hspace{8cm} \textbf{Classe :} \hspace{3cm} \textbf{Date :}\\
\vspace{-0.8cm}
\begin{center}
  \textit{La vie c’est comme une bicyclette, il faut avancer pour ne pas perdre l’équilibre.} - \textbf{Albert Einstein}
\end{center}
\vspace{-0.8cm}

\subsection*{Définition}

\textbf{Un tableau de proportionnalité est}\dotfill \\ \Pointilles[3] 

\textbf{Ex1 : Les triangles existent-il ?}

\textit{Les tableaux sont-ils proportionnels ? } 

\begin{multicols}{2}\noindent
  \begin{center}\begin{tabular}{|c|c|c|} \hline
    24 & 96 & 212 \\  \hline
    2 & 8 & 26\\  \hline
  \end{tabular}\end{center}

  \Pointilles[5]  \columnbreak 

  \begin{center}\begin{tabular}{|c|c|c|} \hline
    12 & 18 & 6 \\  \hline
    30 & 45 & 15\\  \hline
  \end{tabular}\end{center}

  \Pointilles[5] 

\end{multicols}

\subsection*{Ex 1 - Calculer}

\begin{multicols}{4}\noindent
  \begin{center} \begin{tabular}{|c|c|}  \hline
      6    & 32\\  \hline
      14 & $\phantom{\dfrac{azertyuiop}{O}}$\\  \hline
    \end{tabular} \end{center}
  \Pointilles[2]
  \begin{center} \begin{tabular}{|c|c|}   \hline
      12  & 28\\  \hline
      32 & $\phantom{\dfrac{azertyuiop}{O}}$\\  \hline
    \end{tabular} \end{center}
  \Pointilles[2]
  \begin{center} \begin{tabular}{|c|c|}   \hline
      14  & 24\\  \hline
      80 & $\phantom{\dfrac{azertyuiop}{O}}$\\  \hline
    \end{tabular} \end{center}
  \Pointilles[2]
  \begin{center}\begin{tabular}{|c|c|}  \hline
      12 & 46\\  \hline
      $\phantom{\dfrac{azertyuiop}{O}}$ & 24\\  \hline
    \end{tabular} \end{center}

  \Pointilles[2]

\end{multicols}

\begin{center} \begin{tabular}{|c|c|c|c|c|c|} \hline
   3 &  3,1                   &                  45000 &  $\phantom{\dfrac{azertyuiop}{O}}$ & $\phantom{\dfrac{azertyuiop}{O}}$&                     9\\ \hline
   7 &  $\phantom{\dfrac{azertyuiop}{O}}$ & $\phantom{\dfrac{azertyuiop}{O}}$ &                   7400 &                    0,4 &  $\phantom{\dfrac{azertyuiop}{O}}$\\ \hline     
  \end{tabular}\end{center}

\Pointilles[5]


\newpage

\subsection*{Problème 1}

\begin{multicols}{2}\noindent

Un cinéma propose les tarifs suivants. \\
Le prix est-il proportionnel au nombre de séance ?


\begin{center}\begin{tabular}{|c|c|c|c|} \hline
  Nombre de séances & 1 &  4 & 12 \\  \hline
  Prix à payer      & 7 & 28 & 80\\  \hline
\end{tabular}\end{center}  \columnbreak 

\Pointilles[5] 

\end{multicols}

\subsection*{Problème 2}

Pour préparer du foie gras, on doit préalablement saupoudrer le foie frais d'un mélange de sel et de poivre. \\
Compléter le tableau et écrire les calculs.

\begin{center}\begin{tabular}{|c|c|c|c|c|c|c|} \hline
  Poivre (en g) & 10 &  $\phantom{\dfrac{azertyuiop}{O}}$  &   $\phantom{\dfrac{azertyuiop}{O}}$ & 12 &  $\phantom{\dfrac{azertyuiop}{O}}$  &  $\phantom{\dfrac{azertyuiop}{O}}$  \\  \hline
  Sel (en g)    & 30 & 60 & 36 &  $\phantom{\dfrac{azertyuiop}{O}}$  & 90 & 75 \\  \hline
\end{tabular}\end{center}

\Pointilles[5] 

\subsection*{Problème 3}

Voici la recette d'un gratin de patates pour \textbf{6 personnes}. 

\begin{minipage}[t]{0.45\textwidth}
  \textbf{Ingrédients : }
  \begin{itemize}
    \item 1,5 kg de patates.
    \item 100 g de fromage râpé.
    \item 60 ml de lait.
    \item 50g de crème fraîche.
  \end{itemize}

\end{minipage}
\begin{minipage}[t]{0.5\textwidth}
  \textbf{Recette :}
  \begin{itemize}
    \item Éplucher les patates et les couper en fines rondelles.
    \item Mélanger le lait et la crème dans un saladier et faites tremper les rondelles de patates dedans.
    \item Disposer des couches successives de rondelles de patates dans un plat à gratin. 
    \item Faire cuire au four pendant 1h, thermostat 7.
  \end{itemize}
\end{minipage}

\begin{enumerate}
  \item[1.] On souhaite adapter la quantité d'ingrédient à \textbf{10 personnes.} Donner la quantité de chaque ingrédient. \\ \Pointilles[7] 
  \item[2.] La température du four va -t-elle être augmenter de façon proportionnel pour cette cuisson pour 10 personnes ?  \\ \Pointilles[2] 
  \item[3.] Un thermostat de four est proportionnel à la température. On donne thermostat 1 = 30°C. Quelle est la température du four thermostat 7. \\ \Pointilles[3] 
\end{enumerate}

\newpage


\textbf{Nom, Prénom :} \hspace{8cm} \textbf{Classe :} \hspace{3cm} \textbf{Date :}\\
\vspace{-0.8cm}
\begin{center}
  \textit{La vie c’est comme une bicyclette, il faut avancer pour ne pas perdre l’équilibre.} - \textbf{Albert Einstein}
\end{center}
\vspace{-0.8cm}

\subsection*{Définition}

\textbf{Un tableau de proportionnalité est}\dotfill \\ \Pointilles[3] 

\subsection*{Ex 1 - Calculer}


\begin{multicols}{4}\noindent
  \begin{center} \begin{tabular}{|c|c|}  \hline
      8    & 40\\  \hline
      12 & $\phantom{\dfrac{azertyuiop}{O}}$\\  \hline
    \end{tabular} \end{center}
  \Pointilles[2]
  \begin{center} \begin{tabular}{|c|c|}   \hline
      16  & 26\\  \hline
      45 & $\phantom{\dfrac{azertyuiop}{O}}$\\  \hline
    \end{tabular} \end{center}
  \Pointilles[2]
  \begin{center} \begin{tabular}{|c|c|}   \hline
      26  & 56\\  \hline
      60 & $\phantom{\dfrac{azertyuiop}{O}}$\\  \hline
    \end{tabular} \end{center}
  \Pointilles[2]
  \begin{center}\begin{tabular}{|c|c|}  \hline
      14 & 35\\  \hline
      $\phantom{\dfrac{azertyuiop}{O}}$ & 75\\  \hline
    \end{tabular} \end{center}

  \Pointilles[2]

\end{multicols}

\begin{center} \begin{tabular}{|c|c|c|c|c|c|} \hline
   3 &  3,1                   &                  45000 &  $\phantom{\dfrac{azertyuiop}{O}}$ & $\phantom{\dfrac{azertyuiop}{O}}$&                     9\\ \hline
   8 &  $\phantom{\dfrac{azertyuiop}{O}}$ & $\phantom{\dfrac{azertyuiop}{O}}$ &                   7400 &                    0,4 &  $\phantom{\dfrac{azertyuiop}{O}}$\\ \hline     
  \end{tabular}\end{center}

\Pointilles[5]

\subsection*{Ex2 - Démontrer ?}

\textit{Les tableaux sont-ils proportionnels ? } 

\begin{multicols}{2}\noindent
  \begin{center}\begin{tabular}{|c|c|c|} \hline
    24 & 96 & 310 \\  \hline
    2 & 8 & 26\\  \hline
  \end{tabular}\end{center}

  \Pointilles[5]  \columnbreak 

  \begin{center}\begin{tabular}{|c|c|c|} \hline
    12 & 18 & 6 \\  \hline
    30 & 45 & 15\\  \hline
  \end{tabular}\end{center}

  \Pointilles[5] 

\end{multicols}

\newpage

\subsection*{Problème 1}

\begin{multicols}{2}\noindent

Un cinéma propose les tarifs suivants. \\
Le prix est-il proportionnel au nombre de séance ?


\begin{center}\begin{tabular}{|c|c|c|c|} \hline
  Nombre de séances & 1 &  4 & 12 \\  \hline
  Prix à payer      & 7 & 28 & 84\\  \hline
\end{tabular}\end{center}  \columnbreak 

\Pointilles[5] 

\end{multicols}

\subsection*{Problème 2}

Pour préparer du foie gras, on doit préalablement saupoudrer le foie frais d'un mélange de sel et de poivre. \\
Compléter le tableau et écrire les calculs.

\begin{center}\begin{tabular}{|c|c|c|c|c|c|c|} \hline
  Poivre (en g) &  5 &  $\phantom{\dfrac{azertyuiop}{O}}$  &   $\phantom{\dfrac{azertyuiop}{O}}$ & 12 &  $\phantom{\dfrac{azertyuiop}{O}}$  &  $\phantom{\dfrac{azertyuiop}{O}}$  \\  \hline
  Sel (en g)    & 20 & 60 & 36 &  $\phantom{\dfrac{azertyuiop}{O}}$  & 90 & 75 \\  \hline
\end{tabular}\end{center}

\Pointilles[5] 

\subsection*{Problème 3}

Voici la recette d'un gratin de patates pour \textbf{5 personnes}. 

\begin{minipage}[t]{0.45\textwidth}
  \textbf{Ingrédients : }
  \begin{itemize}
    \item 1,4 kg de patates.
    \item 100 g de fromage râpé.
    \item 80 ml de lait.
    \item 60g de crème fraîche.
  \end{itemize}

\end{minipage}
\begin{minipage}[t]{0.5\textwidth}
  \textbf{Recette :}
  \begin{itemize}
    \item Éplucher les patates et les couper en fines rondelles.
    \item Mélanger le lait et la crème dans un saladier et faites tremper les rondelles de patates dedans.
    \item Disposer des couches successives de rondelles de patates dans un plat à gratin. 
    \item Faire cuire au four pendant 1h, thermostat 6.
  \end{itemize}
\end{minipage}

\begin{enumerate}
  \item[1.] On souhaite adapter la quantité d'ingrédient à \textbf{12 personnes.} Donner la quantité de chaque ingrédient. \\ \Pointilles[7] 
  \item[2.] La température du four va -t-elle être augmenter de façon proportionnel pour cette cuisson pour 12 personnes ?  \\ \Pointilles[2] 
  \item[3.] Un thermostat de four est proportionnel à la température. On donne thermostat 1 = 30°C. Quelle est la température du four thermostat 6. \\ \Pointilles[3] 
\end{enumerate}


\end{document}